\documentclass{article}
\usepackage{amsthm}
\usepackage{graphicx}
\usepackage{subfig}
\usepackage{physics}
\graphicspath{ {figures/} }

\title{Monte Carlo Methods}
\author{Caitlin Carnahan}
\begin{document}
This document is intended to serve as personal notes
for self-study of \emph{Monte Carlo Methods for Statistical Physics}
by M.E.J. Newman and G.T. Barkema.

\section{Introduction}
\subsection{Statistical Mechanics}
Many systems of interest are composed of very many simple
constituent systems. For example, a liter of oxygen
is composed of $~10^{22}$ oxygen molecules. The equations
of motion for a single oxygen molecule are relatively simple
but the sheer number of oxygen molecules in the larger system
render the exact solutions infeasible. Instead, we approach
the larger system via probabilistic methods. The goal is to
express the state of the larger system as a set of probabilities
of being in one state or another.

The typical systems studied in physics for which MCM are useful
are systems described by a Hamiltonian $H$ and an
associated energy spectrum that is either continuous or given
by a discrete set $E_{0}$, $E_{1}$, $E_{2}$, etc. We will also
consider systems that interact with a thermal reservoir with
which the system can exchange heat. Heat exchanges with the
reservoir manifest as a small (negligible) perturbation
in the Hamiltonian which pushes the system from one energy
state to another.

Modeling the effect of the perturbation on the Hamiltonian is
done by defining a \emph{dynamics} for the system. The dynamics
of the system may take the form of a set of transition rates.
Suppose the system is in a state $\mu$. Then, $R(\mu \rightarrow \nu)dt$ is
defined to be the probability that the system transitions to a state
$\nu$ in time $dt$, where $R(\mu \rightarrow \nu)$ is defined to be the
\emph{transition rate}. We may also define a set of weights $w_{\mu}(t)$
which represent that probability that the system will be found in state
$\mu$ at time $t$. Using these elements, we can write down a \emph{master
equation}:
\begin{equation}
\frac{dw_{\mu}}{dt} = \sum_{\nu}[w_{\nu}(t)R(\nu \rightarrow \mu)
                              - w_{\mu}(t)R(\mu \rightarrow \nu)]
\end{equation}
The first term in the sum is the rate at which the system is
transitioning into state $\mu$. The second term is the rate at which
the system is transitioning out of state $\mu$.

We may define the expectation value of some observable $Q$, which
takes the value $Q_{\mu}$ in state $\mu$, as
$$ \left \langle Q \right \rangle = \sum_{\mu} Q_{\mu}w_{\mu}(t) $$
The expectation value of $Q$ can be interpreted two ways:
\begin{itemize}
\item It is the mean of the observed value of $Q$ that would be
obtained if we measured many identically prepared systems simultaneously.
\item It is the time average of the quantity $Q$. That is, it might be the
average observed value of $Q$ after many measurements over time.
\end{itemize}

\subsection{Equilibrium}
An \emph{equilibrium} state can be defined as a state in which all of the
rates of change $dw_{\mu}/dt$ vanish, and therefore the transition rate
into and out of each state $\mu$ must be equal. All systems goverened
by the master equation must come to equilibrium after some time and so
we can define the equilibrium values of the weights, the
\emph{equilibrium occupation probabilities}, as
$$p_{\mu} = \lim_{t\rightarrow \infty} w_{\mu}(t)$$
For a system in equilibrium with thermal reservoir at temperature T, the
Boltzmann distribution is
$$p_{\mu} = \frac{1}{Z}e^{-E_{\mu}/kT} = \frac{1}{Z}e^{-\beta E_{\mu}}$$
where $Z$ is the normalization constant, called the \emph{partition function}, and
defined as
$$Z = \sum_{\mu}e^{-\beta E_{\mu}}$$
Plugging this into our definition of expectation values gives us
$$ \left \langle Q \right \rangle = \sum_{\mu} Q_{\mu}p_{\mu}(t)
                                  = \frac{1}{Z}\sum_{\mu} Q_{\mu}e^{-\beta E_{\mu}} $$
which is the expression for the expectation value of a system in equilibrium.
For example, the internal energy $U$ of a system is given by
$$U = \left \langle E \right \rangle
                                  = \frac{1}{Z}\sum_{\mu} E_{\mu}e^{-\beta E_{\mu}}
                                  = -\frac{1}{Z}\frac{\partial Z}{\partial \beta}
                                  = -\frac{\partial log Z}{\partial \beta}$$
It is sometimes appropriate in MCM calculations to calculate the partition function, from which
many other quantities can be derived. Some examples can be found in Section 1.2 of Newman and
Barkema.

\subsection{Fluctuations, Correlations, and Responses}
It is useful to assess the fluctuations in observable quantities. Doing so
allows us to determine how the quantity we are measuring varies over time and
therefore how much of an approximation is being made by calculating the expectation
value. As an example, the root-mean-square fluctuation in internal energy is given by
$$\sqrt{\left \langle E^{2} \right \rangle - \left \langle E \right \rangle^{2}} = \sqrt{\frac{\partial^{2}log Z}{\partial \beta^{2}}}$$
In the limit of a large system, called the \emph{thermodynamic limit}, it is common for
fluctuations to become negligible. For example, RMS fluctuations in internal energy scale
as $\sqrt{V}$ in a system of volume $V$, but the internal energy itself scales as $V$
so as the system becomes very large, the fluctuations grow increasingly
negligible.

Every parameter that can be fixed in a system (e.g. volume, external field)
has a conjugate variable (e.g. pressure, magnetization) that is given by a
derivative of the free energy. For example,
$$ p = -\frac{\partial F}{\partial V}$$
$$ M = \frac{\partial F}{\partial B}$$
The Hamiltonian contains terms of the form $-XY$ where $Y$ is a "field" whose
value is fixed, and $X$ is the conjugate variable to which it couples.
A common trick to calculate the thermal average of a quantity is the following:
\begin{itemize}
\item Make up a fictitious field that couples to the quantity of interest.
\item Add an appropriate term to the Hamiltonian.
\item Set the field to zero after performing the derivative on the free energy.
\end{itemize}
The \emph{susceptibility} of $X$ to $Y$ is a measurement of the strength of
the response of $X$ to variation in $Y$. Typically, this quantity is denoted
by $\chi$ and can be defined as
$$\chi = \frac{\partial \left \langle X \right \rangle  }{\partial Y}$$
The \emph{linear response theorem} tell us that the fluctuations in a variable
are proportional to the susceptibility of the variable to its conjugate field.

\subsubsection{Correlation Functions}
The \emph{disconnected correlation function} $G^{(2)}(i,j)$ is defined as
$$G^{(2)}(i,j) = \left \langle x_{i}x_{j} \right \rangle$$
This function gives us roughly an idea of how the variables $x_{i}$ and $x_{j}$
are correlated. Consider the following scenarios:
\begin{itemize}
\item If $x_{i}$ and $x_{j}$ move (roughly) together most of the time around
the origin zero, then the average of their product will be a positive value.
\item If $x_{i}$ and $x_{j}$ move (roughly) together but in opposite dirrections
most of the time, then the average of their product will be a negative value.
\item If $x_{i}$ and $x_{j}$ move with no discernible pattern relative to one
another, then their product will be sometimes positive and sometimes negative,
averaging to zero over a long time.
\end{itemize}
\end{document}
